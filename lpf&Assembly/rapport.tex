\documentclass{article}
\usepackage[french]{babel}
\usepackage[letterpaper,top=2cm,bottom=2cm,left=3cm,right=3cm,marginparwidth=1.75cm]{geometry}
\usepackage{amsmath}
\usepackage{graphicx}
\usepackage[colorlinks=true, allcolors=blue]{hyperref}

\title{Compilation d’expressions arithmétiques}
\author{Evelyne Le Bezvoët}

\begin{document}
\maketitle


\section{Problèmes rencontrés lors du projet}

Au début du projet, j'ai essayé de coder un lexeur à la main, mais j'ai vu que je mettais beaucoup trop de temps à corriger des bugs. J'ai donc pris la décision d'utiliser ocamllex et ocamlyacc. J'ai encore eu beaucoup de problèmes de compilation: pour l'expression 1+2, mon lexer reconnaissait Int(+1) INT(+2) et mon parser affichait une erreur. J'ai dû changer la définition des entiers dans le lexeur.

Le lexeur et le parseur ne reconnaissent pas l'expression -.1 , puisqu'il y a ambiguïté entre "PLUSF INT(1)" qui est faux et "PLUS FLOAT(0.1)". Je n'ai pas encore réussi à résoudre ce problème.

Je n'ai pas encore fait les conversions int() et float().

Sinon aritha marche bien sur les entiers.
J'ai tenté l'addition sur les flottants, mais mon algorithme beugue encore.

\end{document}
